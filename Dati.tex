%----------------------------------------------------------------------------------------
%   USEFUL COMMANDS
%----------------------------------------------------------------------------------------

\newcommand{\dipartimento}{Dipartimento di Matematica ``Tullio Levi-Civita''}

%----------------------------------------------------------------------------------------
% 	USER DATA
%----------------------------------------------------------------------------------------

% Data di approvazione del piano da parte del tutor interno; nel formato GG Mese AAAA
% compilare inserendo al posto di GG 2 cifre per il giorno, e al posto di
% AAAA 4 cifre per l'anno
\newcommand{\dataApprovazione}{Data}

% Dati dello Studente
\newcommand{\nomeStudente}{Timoty}
\newcommand{\cognomeStudente}{Granziero}
\newcommand{\matricolaStudente}{1123442}
\newcommand{\emailStudente}{timoty.granziero@studenti.unipd.it}
\newcommand{\telStudente}{+ 39 347 5917271}

% Dati del Tutor Aziendale
\newcommand{\nomeTutorAziendale}{Fabio}
\newcommand{\cognomeTutorAziendale}{Pallaro}
\newcommand{\emailTutorAziendale}{f.pallaro@synclab.it}
\newcommand{\telTutorAziendale}{+ 049 8171060}
\newcommand{\ruoloTutorAziendale}{Senior Consultant}

% Dati dell'Azienda
\newcommand{\ragioneSocAzienda}{Sync Lab srl}
\newcommand{\indirizzoAzienda}{Galleria Spagna, 28, Padova (PD)}
\newcommand{\sitoAzienda}{https://www.synclab.it}
\newcommand{\emailAzienda}{info@synclab.it}
\newcommand{\partitaIVAAzienda}{P.IVA 07952560634}

% Dati del Tutor Interno (Docente)
\newcommand{\titoloTutorInterno}{Prof.}
\newcommand{\nomeTutorInterno}{Gilberto}
\newcommand{\cognomeTutorInterno}{Filè}

\newcommand{\prospettoSettimanale}{
     % Personalizzare indicando in lista, i vari task settimana per settimana
     % sostituire a XX il totale ore della settimana
    \begin{itemize}
        \item \textbf{Prima Settimana}
        \begin{itemize}
            \item Presentazione strumenti di lavoro per la condivisione del materiale di studio e per la gestione
                dell'avanzamento del percorso (Slack, Asana, ecc\dots);
            \item Condivisione scaletta di argomenti;
            \item Veloce panorama su metodologie Agile/Scrum;
            \item Studio architettura a micro-servizi.
        \end{itemize}
        \item \textbf{Seconda Settimana}
        \begin{itemize}
            \item Approfondimenti del linguaggio procedurale MongoDB;
            \item Java Standard Edition - ripasso generale.
        \end{itemize}
        \item \textbf{Terza Settimana}
        \begin{itemize}
            \item Java Enterprise Edition: JSP/Servlet;
            \item Spring Boot;
            \item Spring Core/MVC/Web.
        \end{itemize}
        \item \textbf{Quarta Settimana}
        \begin{itemize}
            \item Java Enterprise Edition: Spring WS per WebService, SpringData MongoDB.
        \end{itemize}
        \item \textbf{Quinta Settimana}
        \begin{itemize}
            \item Front-end web: Javascript/Typescript in Angular.
        \end{itemize}
        \item \textbf{Sesta Settimana}
        \begin{itemize}
            \item Analisi dei requisiti richiesti dal cliente e degli impatti sull'applicazione di studio;
            \item Implementazione delle modifiche richieste.
        \end{itemize}
        \item \textbf{Settima Settimana}
        \begin{itemize}
            \item Implementazione delle modifiche richieste.
        \end{itemize}
        \item \textbf{Ottava Settimana}
        \begin{itemize}
            \item Conclusione dell'implementazione richiesta;
            \item Verifica dell'intervento - collaudo finale;
            \item Consegna software e messa in esercizio.
        \end{itemize}
    \end{itemize}
}

% Indicare il totale complessivo (deve essere compreso tra le 300 e le 320 ore)
\newcommand{\totaleOre}{300}

\newcommand{\obiettiviObbligatori}{
	 \item \underline{\textit{O01}}: Acquisizione competenze sulle tematiche sopra descritte;
	 \item \underline{\textit{O02}}: Capacità di raggiungere gli obiettivi richiesti in autonomia seguendo il cronoprogramma;
	 \item \underline{\textit{O03}}: Portare a termine le modifiche richieste dal cliente con una percentuale di superamento pari al 50\%.
}

\newcommand{\obiettiviDesiderabili}{
    \item \underline{\textit{D01}}: Portare a termine le modifiche richieste dal cliente con una percentuale di superamento pari all'80\%.
}

\newcommand{\obiettiviFacoltativi}{
	 \item \underline{\textit{F01}}: Acquisizione competenze sul framework Spring Cloud.
}
